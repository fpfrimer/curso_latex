% Template for Carleton student presentations
% Author: Andrew Gainer-Dewar, 2013
% This work is licensed under the Creative Commons Attribution 4.0 International License.
% To view a copy of this license, visit http://creativecommons.org/licenses/by/4.0/ or send a letter to Creative Commons, 444 Castro Street, Suite 900, Mountain View, California, 94041, USA.
\documentclass{beamer}

% Beamer has facilities to format handouts and notes to go along with your slides,
% but we won't use them here.
\mode<presentation>

% Beamer comes with *lots* of themes. Boadilla is a nice choice;
% it's elegant and stays out of the way.
% Many of the standard themes are very busy, with tables of contents in the
% sidebar and other messiness. This is usually just a distraction for your audience.
% (If they're looking at the ToC, it's because they're bored!)
\usetheme{Boadilla}

% We then load in the custom Carleton color scheme.
\usecolortheme{carl}

% Finally, we load the style file, which sets up various things for us.
\usepackage{ccpres}

% The lxfonts package is a nice choice for a presentation.
% lmodern and arev are also good choices (but the latter looks a
% bit clunky with math).
\usepackage{lxfonts}

%pacotes para português
\usepackage[utf8]{inputenc}
\usepackage[T1]{fontenc}

\usepackage{hyperref}
\usepackage{verbatim}
\usepackage{graphicx}
\usepackage{keystroke}
\usepackage[newfloat]{minted}

% Set your title and author data here.
% The optional arguments will be used in the running footer;
% simply remove them if you want to re-use the long versions.
\title[Curso de \LaTeX]{Elaboração de Trabalhos Acadêmicos utilizando LaTeX}
\author[F. Pfrimer]{Prof. Felipe Walter Dafico Pfrimer}
\institute[UTFPR]{Universidade Tecnológica Federal do Paraná}
\date{01/03/2021}

\begin{document}

% Each slide goes in a frame environment.
% The first one should be your title page, which is
% set up automatically by Beamer.
\begin{frame}
  \titlepage

  \begin{center}
    Uma visão básica de \LaTeX{} abordada de forma prática.
  \end{center}
\end{frame}

% Frame 1: o que latex?
\begin{frame}
  \frametitle{\LaTeX{} - o que é?}
  \pause
  \begin{block}{\textbf{Tex} --> Donald Knuth}
    Sistema de tipografia --> Criação gráfica 
  \end{block}
  \pause
  \begin{block}{\LaTeX --> Leslie Lamport}
    \begin{itemize}
        \pause
        \item  É um sistema de preparação de documentos que utiliza macros Tex;
        \pause
        \item  Precisa de um compilador;
        \pause
        \item  Diferente dos processadores de texto convencionais: WYSIWYG (``\textit{what you see is what you get}'').
    \end{itemize}
  \end{block}
\end{frame}

% Frame 2: instruções de instalação
\begin{frame}
    \frametitle{\LaTeX{} no windows}
    \pause
    \begin{block}{1 - Instalação do MikTex ou TexLive - Compilador}
        \url{https://miktex.org/} -- \url{https://www.tug.org/}
    \end{block}
    \pause
    \begin{block}{2 - Instalação do GhostView - Visualizador}
        \url{http://pages.cs.wisc.edu/~ghost/}
    \end{block}
    \pause    
    \begin{block}{3 - Instalação do Ghostscript - ferramenta para geração de pdf}
        \url{http://www.ghostscript.com/download/}
    \end{block}
    \pause    
    \begin{block}{4 - Instalação de um editor \LaTeX{}}
        \url{http://www.texniccenter.org/} -- \url{https://www.texstudio.org/}
    \end{block}
    \pause    
    \begin{block}{5 - Baixe um modelo ou crie o seu}
        Procure ajuda on-line!
    \end{block}
\end{frame}

% Frame 3: onde procurar ajuda 

\begin{frame}{Onde procurar ajuda on-line}

    \begin{block}{Sites sobre \LaTeX{}}
        \begin{itemize}
            \item \url{https://ctan.org}
            \item \url{https://www.latex-project.org/}
            \item \url{https://www.overleaf.com/}
            \item \url{https://tex.stackexchange.com/}
            \item google
        \end{itemize}
    \end{block}
\end{frame}

% Frame 4: overleaf

\begin{frame}{Overleaf}
    
    \begin{figure}
        \centering
        \includegraphics[width = 0.2\textwidth]{Figuras/overleaf_og_logo.png}
        %\caption{Caption}
        \label{fig:my_label}
    \end{figure}
    \centering \url{https://www.overleaf.com/} - editor on-line, compilador e visualizador
    \vspace{2cm}
    
    \centering Crie uma conta! Vamos fazer os códigos no Overleaf.
    
\end{frame}

% Frame 5: primeiro documento

\begin{frame}[fragile]
    \frametitle{Primeiro documento}
    
    Crie um novo projeto no Overleaf e copie o código a seguir:
    
    \begin{block}{Primeiro código \LaTeX{}}
        \verbatiminput{Codigos/cod1.tex}    
    \end{block}
    
\end{frame}


\begin{frame}
    \frametitle{Primeira tarefa}
    \begin{enumerate}
        \item Crie um novo projeto no Overleaf e faça três parágrafos;
        \item Copie um texto em português (coloque palavras com acentuação);
        \item Observação: Dois ``\textit{Enters}'' para um novo parágrafo (\Return).
    \end{enumerate}
    \begin{block}{Atividade 1}
        \verbatiminput{Codigos/cod2.tex}    
    \end{block}
\end{frame}

\begin{frame}
    \frametitle{Primeira tarefa - resultado}
    \begin{enumerate}
        \item Algo não está certo!
        \item O que aconteceu com as palavras acentuadas?
    \end{enumerate}
    
    \vspace{2cm}
    
    \centering Vamos entender melhor!
    
\end{frame}

\begin{frame}[fragile]
    \frametitle{Estrutura do código}

    Existem duas partes:
    
    \begin{block}{Preambulo}
        Definição do tipo de documento, entrada de pacotes e configurações. É tudo que deve ser inserido antes de \verb|\begin{document}|.
        
    \end{block}
    
    \begin{block}{Corpo do texto}
        É o texto propriamente dito. É tudo que fica entre \verb|\begin{document}| e \verb|\end{document}|
    \end{block}
    
\end{frame}

\begin{frame}[fragile]
    \frametitle{Estrutura do código - Exemplo}
    
    \inputminted{tex}{Codigos/cod3.tex}  
    
    \centering Note que \% é usado para comentários.
    
    \centering Como resolver o problema dos símbolos da atividade 1?
    
\end{frame}

\begin{frame}[fragile]
    \frametitle{Corrigindo o código da atividade 1}
    
    \inputminted[fontsize=\scriptsize]{tex}{Codigos/cod4.tex}  
    
    \centering Note que foram acrescentados os pacotes babel, inputenc, e fontenc.
    
    \centering Use o pacote \verb|identfirst| para identar o primeiro parágrafo.
    
\end{frame}

\begin{frame}[fragile]
    \frametitle{Capítulos, seções e subseções}
    
    Utilize os comandos:
    \begin{itemize}
        \item \verb|\chapter{}|;
        \item \verb|\section{}|;
        \item \verb|\subsection{}|;
        \item e \verb|\subsubsection{}|
    \end{itemize}
    
    para estruturar o texto.
\end{frame}

\begin{frame}[fragile]
    \frametitle{Capítulos, seções e subseções - Exemplo}
    
    \inputminted[fontsize=\scriptsize]{tex}{Codigos/cod5.tex}
    
\end{frame}

\begin{frame}[fragile]
    \frametitle{Atividade 2}
    
    Crie um documento com três capítulos, duas seções por capítulo e duas subseções em uma das seções. Faça referências cruzadas. 
    
\end{frame}

\begin{frame}[fragile]
    \frametitle{Um documento um pouco mais completo: capa e sumário}
    
    \inputminted[fontsize=\scriptsize]{tex}{Codigos/cod6.tex}
    
\end{frame}

\begin{frame}[fragile]
    \frametitle{A classe Abntex2}
    
    O abnTeX2, evolução do abnTeX (\textit{ABsurd Norms for TeX}), é uma suíte para LaTeX que atende os requisitos das normas da ABNT (Associação Brasileira de Normas Técnicas) para elaboração de documentos técnicos e científicos brasileiros, como artigos científicos, relatórios técnicos, trabalhos acadêmicos como teses, dissertações, projetos de pesquisa e outros documentos do gênero.
    
    Vejamos a página do projeto: \url{https://www.abntex.net.br/}
    
\end{frame}

\begin{frame}[fragile]
    \frametitle{Modelo de trabalho de conclusão de curso}
    
    \begin{itemize}
        \item Baixar em \url{https://github.com/fpfrimer/curso_latex}
        \item Fazer o upload para o overleaf.
    \end{itemize}
    
    \vspace{2cm}
    
    O modelo foi construído com base nos seguintes pacotes:
    \begin{itemize}
        \item abntex2: \url{https://ctan.org/pkg/abntex2}
        \item abntex2cite \url{http://mirrors.ibiblio.org/CTAN/macros/latex/contrib/abntex2/doc/abntex2cite.pdf}
    \end{itemize}
    
\end{frame}



\end{document}